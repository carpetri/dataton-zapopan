\thispagestyle{plain}
\begin{center}
\scshape \Large Datatón: Predicción espacial de delitos en Zapopan\footnote{Reporte ejecutivo del proyecto: Predicción espacial de delitos en Zapopan del equipo PETRUS.
\\ Agradecemos el apoyo y comentarios de todos nuestros colaboradores del programa: \href{http://dssg.uchicago.edu}{\textit{Data Science for Social Good (DSSG)}}.
\\ Esta versión: \today.}

\bigskip
\large \scshape Carlos Petricioli Araiza\footnote{ \href{mailto:carpetri@gmail.com}{\Letter \ carpetri@gmail.com} }  \hfill Christopher Lazarus García
\footnote{ \href{mailto:clazarusg@gmail.com}{\Letter \ clazarusg@gmail.com} } 
\normalsize
\end{center}

\bigskip

\section*{\centering  \begin{normalsize}  Resumen\end{normalsize}}
\begin{quotation}
\noindent 
Con la finalidad de demostrar los beneficios que tiene el análisis de datos y la explotación de grandes volúmenes de información para el mejoramiento de la política pública, el objetivo de este proyecto es hacer una visualización de los delitos en el municipio de Zapopan, Jalisco en el que se pueda observar una predicción razonable de los delitos que ocurriran. Esta predicción se realizará utilizando información sobre centros recreativos, centros de salud, información sobre las colonias, sobre delitos cometidos y delitos detenidos así como información sobre parques, templos, tianguis y obras públicas. Además se consideran factores climatológicos y culturales como festividades y Eventos importantes en el municipio de Zapopan.  El objetivo indirecto de este proyecto es servir de ejemplo de cómo se puede utilizar este tipo de información para el mejoramiento y desarrollo del país a partir de uso de datos abiertos.


\end{quotation}

\vfill
Tópicos: Análisis geoespacial,  Impacto Ambiental, Minería de datos, INEGI, México y Seguridad.

