\thispagestyle{plain}
\begin{center}
\scshape \Large Datatón: Predicción espacial de delitos en Zapopan\footnote{Reporte ejecutivo del proyecto: Predicción espacial de delitos en Zapopan del equipo PETRUS.
\\ Agradecemos el apoyo y comentarios de todos nuestros colaboradores del programa: \href{http://dssg.uchicago.edu}{\textit{Data Science for Social Good (DSSG)}}.
\\ Esta versión: \today.}

\bigskip
\large \scshape Carlos Petricioli Araiza\footnote{ \href{mailto:carpetri@gmail.com}{\Letter \ carpetri@gmail.com} }  \hfill Christopher Lazarus García
\footnote{ \href{mailto:clazarusg@gmail.com}{\Letter \ clazarusg@gmail.com} } 
\normalsize
\end{center}

\bigskip
% Tema de investigación.
% Breve síntesis metodológica.
% Resumen de resultados.
% Potencial de impacto.
% Una visualización que sustente sus resultados.
\section*{\centering  \begin{normalsize} Tema de investigación y Síntesis metodológica. \end{normalsize}}
\begin{quotation}
\noindent 
El tema de este proyecto es la detección y predicción del crimen en el municipo de Zapopan. El primer objetivo de este proyecto es hacer una visualización espacial satisfactoria de los delitos en el municipio de Zapopan, Jalisco con el objetivo de identificar las zonas más conflictivas y así poder proceder a hacer una predicción sobre dónde habrá más delitos. Para hacer una  predicción razonable se utilizan los delitos cometidos y los detenidos. Se controla por el tipo de delito y todo el análisis se realiza utilizando información sobre centros recreativos, centros de salud, información sobre las colonias,  así como información sobre parques, templos, tianguis y obras públicas. Además se consideran factores climatológicos y culturales como festividades y eventos importantes en el municipio de Zapopan.  El objetivo indirecto de este proyecto es servir de ejemplo de cómo se puede utilizar este tipo de información para el mejoramiento y desarrollo del país a partir de uso de datos abiertos.


\end{quotation}

\vfill
Tópicos: Análisis geoespacial,  Impacto Ambiental, Minería de datos, INEGI, México y Seguridad.

